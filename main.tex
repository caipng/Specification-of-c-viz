\documentclass[11pt]{article}
\usepackage[showframe=true]{geometry}
% \usepackage{geometry}
\geometry{a4paper,top=1in,bottom=1in,left=1.2in,right=1.2in}

\usepackage[dvipsnames]{xcolor}

%%%%%%%%%%%%%%%%%%%%%%%%%%%%%%%%% Font setup %%%%%%%%%%%%%%%%%%%%%%%%%%%%%%%%%%
\usepackage{libertine}
\ExplSyntaxOn
\bool_if:nTF { \sys_if_engine_xetex_p: || \sys_if_engine_luatex_p: }
  {
    % Load mathtools before unicode-math
    \usepackage{mathtools}
    \usepackage[
      math-style=ISO,
      warnings-off={mathtools-colon, mathtools-overbracket},
    ]{unicode-math}
    \setmathfont{LibertinusMath-Regular.otf}[Scale = MatchUppercase]
    \setmathfont{Garamond-Math.otf}[
      Scale = MatchUppercase,
      range = {\Coloneq}
    ]

    \setmonofont{Inconsolatazi4}[
      Extension = .otf,
      UprightFont = *-Regular,
      BoldFont = *-Bold,
      Scale=MatchLowercase,
      AutoFakeSlant = 0.2,
    ]
  }
  {
    \usepackage[libertine]{newtxmath}
    \usepackage{inconsolata}
  }
\ExplSyntaxOff

%%%%%%%%%%%%%%%%%%%%%%%%%%%%%%%% Presentations %%%%%%%%%%%%%%%%%%%%%%%%%%%%%%%%
\usepackage{tcolorbox}
\tcbuselibrary{listings,breakable}
\tcbset{listing engine=listings, colframe=black, colback=white, size=small}

\NewTCBListing { example } { !O{} !s }
  {
    sharp corners,
    IfBooleanTF = { #2 }
      { listing above text }
      { listing side text },
    fontlower = \small,
    #1
  }

\NewTCBListing { listing } { !O{} }
  { sharp corners, listing only, #1 }

\NewDocumentEnvironment {exampleside} {}
  { \tcblisting{listing side text,righthand width=.5\textwidth} }
  { \endtcblisting }

\NewDocumentEnvironment { presentcommand } { b }
  {%
    \vspace*{0.5\baselineskip}\noindent\fbox{%
    \begin{minipage}{\dimexpr\textwidth-2\fboxsep-2\fboxrule}
      #1
    \end{minipage}}\vspace*{0.5\baselineskip}
  }
  { }

\NewDocumentCommand \cmd { m } {\texttt{\char`\\#1}}
\NewDocumentCommand \env { m m }
  {
    \texttt{%
      \char`\\begin\{#1\} \textrm{#2}%
      \char`\\end\{#1\}%
    }%
  }


\usepackage{tabularray}
\UseTblrLibrary{booktabs}
% \fakeverb
\usepackage{codehigh}

\newcommand*{\secref}[1]{\hyperref[{#1}]{section~\ref*{#1} -- \emph{\nameref*{#1}}}}
\newcommand*{\Secref}[1]{\hyperref[{#1}]{Section~\ref*{#1} -- \emph{\nameref*{#1}}}}
\newcommand*{\subsecref}[1]{\hyperref[{#1}]{subsection~\ref*{#1} -- \emph{\nameref*{#1}}}}
\newcommand*{\subsubsecref}[1]{\hyperref[{#1}]{\nameref*{#1}}}
\newcommand*{\parref}[1]{\hyperref[{#1}]{\nameref*{#1}}}

%%%%%%%%%%%%%%%%%%%%%%%%%%%%%%%%%% simplebnf %%%%%%%%%%%%%%%%%%%%%%%%%%%%%%%%%%
\usepackage{simplebnf}
\SetBNFLayout{
    colspec = {Q[0mm] Q[r, font=\sffamily] Q[r] Q[l, mode=text, font=\ttfamily, wd=90mm] Q[l]},
    % row{-} = {bg = azure9},
}
\SetBNFConfig{
    prod-delim = ;;;,
    new-line-delim = !,
    single-line-delim = @,
    comment = //,
    relation = {:::=},
    relation-sym-map =
    {
      {:::=} = $\Coloneqq$,
    },
}

%%%%%%%%%%%%%%%%%%%%%%%%%%%%%%%% Miscellaneous %%%%%%%%%%%%%%%%%%%%%%%%%%%%%%%%
\usepackage{hologo}
\newcommand*{\XeLaTeX}{\hologo{XeLaTeX}}
\newcommand*{\LuaLaTeX}{\hologo{LuaLaTeX}}

\usepackage{pdflscape}

% ``Make sure it comes *last* of your loaded packages"
\usepackage[pdfborder={0 0 0}, colorlinks=true]{hyperref}
\hypersetup{
    linkcolor=BrickRed
}

\newcommand{\specsyntax}[2][90mm]{
    \subsubsection*{Syntax}
    \centerline{
        \begin{bnf}[colspec={Q[0mm]Q[r,font=\sffamily]Q[r]Q[l,mode=text,font=\ttfamily,#1]Q[l]}]
        #2
        \end{bnf}
    }
    \bigskip
}

\newcommand{\specconstraints}[1]{
    \subsubsection*{Constraints}
    {
        \setlength{\parindent}{0pt}
        \tiny
        \setlength{\parskip}{\baselineskip}
        \normalsize
        \vspace{-10pt}
        #1
    }
}

\newcommand{\specsem}[1]{
    \subsubsection*{Semantics}
    {
        \setlength{\parindent}{0pt}
        \tiny
        \setlength{\parskip}{\baselineskip}
        \normalsize
        \vspace{-10pt}
        #1
    }
}

\newcommand{\spec}[1]{
    {
        \vspace{8pt}
        \setlength{\parindent}{0pt}
        \tiny
        \setlength{\parskip}{\baselineskip}
        \normalsize
        \vspace{-10pt}
        #1
    }
}

\usepackage{fancyvrb}
\usepackage[framemethod=tikz]{mdframed}

\newcommand{\deviation}[1]{
    \begin{mdframed}[hidealllines=true,backgroundcolor=blue!8,innerleftmargin=3pt,innerrightmargin=3pt,leftmargin=-3pt,rightmargin=-3pt,innertopmargin=3pt,innerbottommargin=3pt]
    #1
    \end{mdframed}
    \vspace{-14pt}
}

\usepackage{tablefootnote} 
\makeatletter 
\AfterEndEnvironment{mdframed}{%
 \tfn@tablefootnoteprintout% 
 \gdef\tfn@fnt{0}% 
}
\makeatother 

% \newcommand*{\mycommentstyle}[1]{%
%   \begingroup
%     \sffamily
%     \itshape
%     \lstset{columns=fullflexible}%
%     #1%
%   \endgroup
% }
\lstset{language=C,xleftmargin=10pt,belowskip=0pt,basicstyle=\ttfamily\footnotesize,commentstyle=\sffamily\itshape,columns=fullflexible}

\usepackage{array}

%%%%%%%%%%%%%%%%%%%%%%%%%%%%%%%%%% Metadata %%%%%%%%%%%%%%%%%%%%%%%%%%%%%%%%%%%
\title{%
  \textsf{c-viz} --- Specification for \texttt{v0.1.1}%
  \footnote{Source code made available at this \href{https://github.com/caipng/c-viz/tree/d1f985e7ef38a52ca8c50dae77cd190145ecdfb2}{public repository}}}
\author{Cai Kai'An\footnote{E-mail: %
  \href{mailto:kaian531@gmail.com}{\texttt{kaian531@gmail.com}}}}
\date{26/04/2024}

\pagestyle{headings}
\renewcommand{\subsectionmark}[1]{%
  \markright{\MakeUppercase{\thesubsection.\ #1}}%
}

%%%%%%%%%%%%%%%%%%%%%%%%%%%%%%%%%% Document %%%%%%%%%%%%%%%%%%%%%%%%%%%%%%%%%%%
\begin{document}

\maketitle

\vfill

\verb|c-viz| is a project that aims to develop a web-based explicit-control evaluator\footnote{As described in \href{https://sourceacademy.org/sicpjs/5.4}{SICP - JS Edition Section 5.4}} for a suitably chosen sublanguage of \verb|C|.
The goal is to create an environment for running \verb|C| programs suitable for first-year computer science students.
The mental model encouraged by the visualizer avoids knowledge of a \verb|C| compiler or intimate details of the underlying hardware, which tend to be obstacles for many learners.
At the same time, students will still be exposed to important concepts such as undefined behaviours and \verb|C|'s memory model.
\par
\medskip
This document specifies the subset of \verb|C| supported.

\begingroup
  \hypersetup{hidelinks}
  \tableofcontents
\endgroup

\bigskip

\newpage
\section*{Introduction}\label{sec:intro}
\addcontentsline{toc}{section}{Introduction}
\markright{INTRODUCTION}

This document follows the structure of (and frequently refers to\footnote{certain definitions and wordings may be lifted \emph{verbatim} from the standard}) the ISO/IEC
9899:1999 - n1256 version of the C99 standard\footnote{\href{https://www.open-std.org/jtc1/sc22/wg14/www/docs/n1256.pdf}{https://www.open-std.org/jtc1/sc22/wg14/www/docs/n1256.pdf}}.
Where applicable, we express grammars using Extended Backus–Naur form\footnote{we use $\Coloneqq$ for definition, \lstinline{,} for concatenation, \lstinline{|} for alternation, $[\ x\ ]$ for optional $x$, $\{\ x\ \}$ for zero or more repetitions of $x$, $(\ \dots\ )$ for grouping, {\textsf `}$\dots${\textsf '} for terminals, $-$ for negation and {\textsf ?}$\dots${\textsf ?} for special sequences.}, adopting the ISO/IEC 14977 standard proposed by R. S. Scowen.
\par
\deviation{
Deviations from the \Verb!C! standard will be on their own paragraph and highlighted as such.
}
\vspace{10pt}
\par
\Secref{sec:concepts} describes various language concepts, such as scopes of identifiers and types.
\Secref{sec:lexgrammar} to \secref{sec:extdef} detail the syntax, type checking constraints and semantics of various language constructs.
In \secref{sec:memory} we look at the the various memory segments and memory related runtime data structures implemented in the interpreter.
Lastly, an informal description of the big step operational semantics of the language is provided in \secref{sec:opsem}.

\newpage
\section{Concepts}\label{sec:concepts}

\subsection{Numerical limits}\label{subsec:numlimits}

The following are the constant values defined for the numerical limits of integer types.
A future version of \Verb|c-viz| may allow for configuration of these values.

\begin{table}[h]
\begin{tabular}{>{\ttfamily \bfseries}r>{$}l<{$}l}
CHAR\_BIT    &  8 &  number of bits for smallest object (byte)\\
SCHAR\_MIN   &  -(2^7-1) &  minimum value for an object of type \Verb!signed char!\\
SCHAR\_MAX   &  2^7 - 1 &  maximum value for an object of type \Verb!signed char!\\
UCHAR\_MAX   &  2^8 - 1 &  maximum value for an object of type \Verb!unsigned char!\\
CHAR\_MIN    &  -(2^7-1) &  minimum value for an object of type \Verb!char!\\
CHAR\_MAX    &  2^7 - 1 &  maximum value for an object of type \Verb!char!\\
MB\_LEN\_MAX &  1 & maximum number of bytes in a multibyte character\\
SHRT\_MIN    &  -(2^{15}-1)&  minimum value for an object of type \Verb!short int!\\
SHRT\_MAX    &  2^{15}-1&  maximum value for an object of type \Verb!short int!\\
USHRT\_MAX   &  2^{16}-1&  maximum value for an object of type \Verb!unsigned short int!\\
INT\_MIN     &  -(2^{31}-1)&  minimum value for an object of type \Verb!int!\\
INT\_MAX     &  2^{31}-1&  maximum value for an object of type \Verb!int!\\
UINT\_MAX    &  2^{32}-1&  maximum value for an object of type \Verb!unsigned int!\\
LONG\_MIN    &  -(2^{31}-1)&  minimum value for an object of type \Verb!long int!\\
LONG\_MAX    &  2^{31}-1&  maximum value for an object of type \Verb!long int!\\
ULONG\_MAX   &  2^{32}-1&  maximum value for an object of type \Verb!unsigned long int!\\
LLONG\_MIN   &  -(2^{63}-1)&  minimum value for an object of type \Verb!long long int!\\
LLONG\_MAX   &  2^{63}-1&  maximum value for an object of type \Verb!long long int!\\
ULLONG\_MAX  &  2^{64}-1& maximum value for an object of type \Verb!unsigned long long int!
\end{tabular}
\end{table}

\subsection{Scopes of identifiers}\label{subsec:scopes}
\spec{
An identifier can denote an object; a function; a tag or a member of a structure; or a typedef name. The same identifier can denote different entities at different points in the program.
\par
For each different entity designated by an identifier, the identifier can only be used within a region of program code called its \emph{scope}. Thus an identifier can designate different entities only if they are in different scopes or in different name spaces.
\par
\deviation{
There are 2 kinds of scopes: file and block. Function prototype scope and function scope\tablefootnote{used only by label names for \Verb!goto! constructs} are not supported.
}
\par

}

\subsection{Types}\label{subsec:types}

\section{Lexical Grammar}\label{sec:lexgrammar}
\section{Expressions}\label{sec:expr}
\section{Declarations}\label{sec:decl}
\section{Statements}\label{sec:stmt}

\newpage
\section{External Definitions}\label{sec:extdef}

\specsyntax{
    TranslationUnit :::=
    {\sffamily ExternalDeclaration} , \{ {\sffamily ExternalDeclaration} \}
    // \hyperref[sec:extdef]{external definitions}
    ;;;
    ExternalDeclaration :::=
    ! {\sffamily FunctionDefinition}
    ! {\sffamily Declaration}
    ;;;
}

\specconstraints{
Every identifier declared in a translation unit can have at most one external definition.
Moreover, if such an identifier is used in an expression, then there shall be exactly one external definition for the identifier in the translation unit.
\par
\deviation{A translation unit must define a \Verb!main! function. The \Verb!main! function shall be defined with no parameters\tablefootnote{\Verb!argc! and \Verb!argv! are not supported} and a return type of \Verb!int!.}
\par
\deviation{Forward declaration of functions are not supported. Functions must be defined prior to any code that references them. On the other hand, forward declaration of \Verb!struct!s, such as to construct self-referential types, are supported in file scope.}
\par
\begin{lstlisting}^^J
void f(int); // ERROR: forward declaration of a function^^J
void f(int a) \{\}^^J
^^J
struct node \{^^J
\ \ int value;^^J
\ \ struct node *next; // OK: forward declaration of struct node^^J
\};^^J
\end{lstlisting}
}

\specsem{
A translation unit is a sequence of external declarations. They are described as ``external" as they appear outside any function. An external definition is an external declaration that is also a definition\footnote{a declaration that causes memory to be reserved for an object or function} of a function or an object.
}

\subsection{Function definitions}\label{subsec:fndef}

\specsyntax{
    FunctionDefinition :::=
    {\sffamily DeclarationSpecifiers} , {\sffamily Declarator} , {\sffamily CompoundStatement}
    // \hyperref[subsec:fndef]{function definitions}
    ;;;
    DeclarationList :::=
    {\sffamily Declaration} , \{ {\sffamily Declaration} \}
}

\specconstraints{
The identifier declared in a function definition shall have function type.
\par
The return type of a function shall be \Verb!void! or an object type other than array type.
\par
\deviation{The declaration specifiers shall not contain any storage-class specifiers.}
\par
The declaration of each parameter shall include an identifier, other than the special case where there is only a single parameter of type \Verb!void! to denote a function that has no parameters.
}

\specsem{
The identifier for each parameter is an lvalue and is effectively declared at the start of the compound statement constituting the function body.
Therefore, these identifiers may not be re-declared in the function body except for in an enclosing block.
\par
\deviation{The layout of the storage for parameters is specified as such: the leftmost argument has address equal to the starting address of the stack frame, and memory allocation proceeds left to right with minimum padding added to ensure alignment.}
\par
On entry to the function, the value of each argument expression is converted to the type of the corresponding parameter as if by assignment.
\par
After all parameters have been assigned, the function body is executed.
\par
\deviation{If the \Verb!\}! that terminates a function is reached, the interpreter throws an error\tablefootnote{functions returning \Verb!void! need to include a \Verb!return! statement as well}. The only exception is in the case of the \Verb!main! function, in which case a value of 0 is returned.}
}

\subsection{External object definitions}\label{subsec:extobjdef}

\specsem{
An external definition for an identifier is a declaration with initializer for an object with file scope.
\par
\deviation{Each identifier may only be declared once. \emph{Tentative definitions} are not supported.}
\par
A declaration without an initializer for an object with file scope behaves as if an initializer exists and is equal to 0.
}

\section{Memory Model}\label{sec:memory}
\section{Operational Semantics}\label{sec:opsem}

\newpage
\appendix
\section{Language Syntax Summary}

Refer to \secref{sec:intro} for an explanation of the notation used.

\subsection{Lexical Grammar}
\medskip
\centerline{
    \begin{bnf}[colspec={Q[0mm]Q[r,font=\sffamily]Q[r]Q[l,mode=text,font=\ttfamily,92mm]Q[l]}]
    Token :::=
    {\sffamily Keyword} @ {\sffamily Identifier} @ {\sffamily Constant}
    ! {\sffamily StringLiteral} @ {\sffamily Punctuator};;;
    \_ :::=
    \{ {\sffamily WhiteSpace} @ {\sffamily LongComment} @ {\sffamily LineComment} \}
    // token separator
    ;;;
    WhiteSpace :::=
    \lstinline{' '} @ \lstinline{'\\n'} @ \lstinline{'\\r'} @ \lstinline{'\\t'} @ \lstinline{'\\u000b'} @ \lstinline{'\\f'};;;
    LongComment :::=
    \lstinline{'\/*'} , \{ {\it ? UTF-16 char ?} - \lstinline{'*\/'} \} , \lstinline{'*\/'};;;
    LineComment :::=
    \lstinline{'\/\/'} , \{ {\it ? UTF-16 char ?} - \lstinline{'\\n'} \};;;
    Keyword :::=
    \lstinline{'auto'} @ \lstinline{'break'} @ \lstinline{'case'} @ \lstinline{'char'} @ \lstinline{'const'} @ \lstinline{'continue'} @ \lstinline{'default'} @ \lstinline{'do'} @ \lstinline{'double'} @ \lstinline{'else'} @ \lstinline{'enum'} @ \lstinline{'extern'} @ \lstinline{'float'} @ \lstinline{'for'} @ \lstinline{'goto'} @ \lstinline{'if'} @ \lstinline{'inline'} @ \lstinline{'int'} @ \lstinline{'long'} @ \lstinline{'register'} @ \lstinline{'restrict'} @ \lstinline{'return'} @ \lstinline{'short'} @ \lstinline{'signed'} @ \lstinline{'sizeof'} @ \lstinline{'static'} @ \lstinline{'struct'} @ \lstinline{'switch'} @ \lstinline{'typedef'} @ \lstinline{'union'} @ \lstinline{'unsigned'} @ \lstinline{'void'} @ \lstinline{'volatile'} @ \lstinline{'while'} @ \lstinline{'\_Alignas'} @ \lstinline{'\_Alignof'} @ \lstinline{'\_Atomic'} @ \lstinline{'\_Bool'} @ \lstinline{'\_Complex'} @ \lstinline{'\_Generic'} @ \lstinline{'\_Imaginary'} @ \lstinline{'\_Noreturn'} @ \lstinline{'\_Static\_assert'} @ \lstinline{'\_Thread\_local'};;;
    Identifier :::=
    {\sffamily IdentifierNondigit} , \{ {\sffamily IdentifierNondigit} @ {\sffamily Digit} \};;;
    IdentifierNondigit :::= {\sffamily Nondigit} @ {\sffamily UniversalCharacterName}
    ;;;
    Nondigit :::=
    {\it ? lowercase or uppercase alphabet ?} @ \lstinline{'\_'}
    ;;;
    Digit :::=
    {\it ? digit from 0 to 9 ?}
    ;;;
    UniversalCharacterName :::=
    ! \lstinline{'\\u'} , {\sffamily HexQuad}
    ! \lstinline{'\\U'} , {\sffamily HexQuad} , {\sffamily HexQuad}
    ;;;
    HexQuad :::=
    {\sffamily HexadecimalDigit} , {\sffamily HexadecimalDigit} , {\sffamily HexadecimalDigit} , {\sffamily HexadecimalDigit};;;
    Constant :::=
    {\sffamily IntegerConstant} @ {\sffamily CharacterConstant};;;
    IntegerConstant :::=
    ( {\sffamily DecimalConstant} @ {\sffamily HexadecimalConstant} @ {\sffamily OctalConstant} ) , [ {\sffamily IntegerSuffix} ];;;
    DecimalConstant :::=
    {\sffamily NonzeroDigit} , \{ {\sffamily Digit} \};;;
    OctalConstant :::=
    \lstinline{'0'} , \{ {\sffamily OctalDigit} \};;;
    HexadecimalConstant :::=
    {\sffamily HexadecimalPrefix} , {\sffamily HexadecimalDigit} , \newline
    \{ {\sffamily HexadecimalDigit} \};;;
    HexadecimalPrefix :::=
    \lstinline{'0x'} @ \lstinline{'0X'}
    ;;;
    NonzeroDigit :::=
    {\it ? digit from 1 to 9 ?}
    ;;;
    OctalDigit :::=
    {\it ? digit from 0 to 7 ?}
    ;;;
    HexadecimalDigit :::=
    ! {\it ? digit from 0 to 9 ?}
    ! {\it ? lowercase or uppercase alphabet from a to f ?}
    ;;;
    IntegerSuffix :::=
    ! {\sffamily UnsignedSuffix} , [ {\sffamily LongLongSuffix} @ {\sffamily LongSuffix} ]
    ! ( {\sffamily LongLongSuffix} @ {\sffamily LongSuffix} ) , [ {\sffamily UnsignedSuffix} ]
    ;;;
    UnsignedSuffix :::=
    \lstinline{'u'} @ \lstinline{'U'};;;
    LongSuffix :::=
    \lstinline{'l'} @ \lstinline{'L'};;;
    LongLongSuffix :::=
    \lstinline{'ll'} @ \lstinline{'LL'}
    \end{bnf}
}

\centerline{
    \begin{bnf}[colspec={Q[0mm]Q[r,font=\sffamily]Q[r]Q[l,mode=text,font=\ttfamily,92mm]Q[l]}]
    CharacterConstant :::=
    \lstinline{'''} , {\sffamily CChar} , \lstinline{'''};;;
    CChar :::=
    ! {\sffamily EscapeSequence}
    ! {\it ? UTF-16 char ?} - ( \lstinline{'''} @ \lstinline{'\\n'} @ \lstinline{'\\'} )
    ;;;
    EscapeSequence :::=
    ! {\sffamily SimpleEscapeSequence} @ {\sffamily OctalEscapeSequence} @ {\sffamily HexadecimalEscapeSequence} @ {\sffamily UniversalCharacterName}
    ;;;
    SimpleEscapeSequence :::=
    \lstinline{'\\'} , ( \lstinline{'''} @ \lstinline{'"'} @ \lstinline{'\?'} @ \lstinline{'\\'} @ \lstinline{'a'} @ \lstinline{'b'} @ \lstinline{'f'} @ \lstinline{'n'} @ \lstinline{'r'} @ \lstinline{'t'} @ \lstinline{'v'} );;;
    OctalEscapeSequence :::=
    \lstinline{'\\'} , {\sffamily OctalDigit} , [ {\sffamily OctalDigit} ] , [ {\sffamily OctalDigit} ];;;
    HexadecimalEscapeSequence :::=
    \lstinline{'\\x'} , {\sffamily HexadecimalDigit} , \{ {\sffamily HexadecimalDigit} \};;;
    StringLiteral :::=
    \lstinline{'"'} , [ {\sffamily SCharSequence} ] , \lstinline{'"'};;;
    SCharSequence :::=
    {\sffamily SChar} , \{ {\sffamily SChar} \};;;
    SChar :::=
    ! {\sffamily EscapeSequence}
    ! {\it ? UTF-16 char ?} - ( \lstinline{'"'} @ \lstinline{'\\n'} @ \lstinline{'\\'} )
    ;;;
    Punctuator :::=
    \lstinline{'['} @ \lstinline{']'} @ \lstinline{'('} @ \lstinline{')'} @ \lstinline{'\{'} @ \lstinline{'\}'} @ \lstinline{'.'} @ \lstinline{'->'} @ \lstinline{'++'} @ \lstinline{'--'} @ \lstinline{'\&'} @ \lstinline{'*'} @ \lstinline{'+'} @ \lstinline{'-'} @ \lstinline{'~'} @ \lstinline{'\!'} @ \lstinline{'\/'} @ \lstinline{'\%'} @ \lstinline{'<<'} @ \lstinline{'>>'} @ \lstinline{'<'} @ \lstinline{'>'} @ \lstinline{'<='} @ \lstinline{'>='} @ \lstinline{'=='} @ \lstinline{'\!='} @ \lstinline{'^'} @ \lstinline{'|'} @ \lstinline{'\&\&'} @ \lstinline{'||'} @ \lstinline{'\?'} @ \lstinline{':'} @ \lstinline{';'} @ \lstinline{'...'} @ \lstinline{'='} @ \lstinline{'*='} @ \lstinline{'\/='} @ \lstinline{'\%='} @ \lstinline{'+='} @ \lstinline{'-='} @ \lstinline{'<<='} @ \lstinline{'>>='} @ \lstinline{'\&='} @ \lstinline{'^='} @ \lstinline{'|='} @ \lstinline{','} @ \lstinline{'\#'} @ \lstinline{'\#\#'} @ \lstinline{'<:'} @ \lstinline{':>'} @ \lstinline{'<\%'} @ \lstinline{'\%>'} @ \lstinline{'\%:'} @ \lstinline{'\%:\%:'}
    \end{bnf}
}

\bigskip
\subsection{Expressions}
\bigskip
\centerline{
    \begin{bnf}[colspec={Q[0mm]Q[r,font=\sffamily]Q[r]Q[l,mode=text,font=\ttfamily,92mm]Q[l]}]
    PrimaryExpression :::=
    ! {\sffamily Identifier}
    ! {\sffamily Constant}
    ! {\sffamily StringLiteral}
    ! \lstinline{'('} , {\sffamily Expression} , \lstinline{')'}
    ;;;
    PostfixExpression :::=
    {\sffamily PrimaryExpression} , \{ {\sffamily PostfixOp} \};;;
    PostfixOp :::=
    ! \lstinline{'['} , {\sffamily Expression} , \lstinline{']'} // array subscripting
    ! \lstinline{'('} , [ {\sffamily ArgumentExpressionList} ] , \lstinline{')'}  // function call
    ! \lstinline{'.'} , {\sffamily Identifier} // struct member
    ! \lstinline{'->'} , {\sffamily Identifier} 
    ! \lstinline{'++'} 
    @ \lstinline{'--'} // postfix incr/decr
    ;;;
    ArgumentExpressionList :::=
    {\sffamily AssignmentExpression} , \newline
    \{ \lstinline{','} , {\sffamily AssignmentExpression} \};;;
    UnaryExpression :::=
    ! {\sffamily PostfixExpression}
    ! \lstinline{'++'} , {\sffamily UnaryExpression} // prefix incr/decr
    ! \lstinline{'--'} , {\sffamily UnaryExpression} 
    ! {\sffamily UnaryOperator} , {\sffamily CastExpression} 
    ! \lstinline{'sizeof'} , \lstinline{'('} , {\sffamily TypeName} , \lstinline{')'} // sizeof operator
    ! \lstinline{'sizeof'} , {\sffamily UnaryExpression} 
    ;;;
    UnaryOperator :::=
    \lstinline{'\&'} // address operator
    ! \lstinline{'*'} // indirection operator
    ! \lstinline{'+'}
    @ \lstinline{'-'}
    @ \lstinline{'~'}
    @ \lstinline{'\!'} // unary arithmetic
    ;;;
    CastExpression :::=
    ! {\sffamily UnaryExpression}
    ! \lstinline{'('} , {\sffamily TypeName} , \lstinline{')'} , {\sffamily CastExpression}
    ;;;
    MultiplicativeExpression :::=
    {\sffamily CastExpression} , \newline
    \{ ( \lstinline{'*'} @ \lstinline{'\/'} @ \lstinline{'\%'} ) , {\sffamily CastExpression} \}
    \end{bnf}
}

\centerline{
    \begin{bnf}[colspec={Q[0mm]Q[r,font=\sffamily]Q[r]Q[l,mode=text,font=\ttfamily,92mm]Q[l]}]
    AdditiveExpression :::=
    {\sffamily MultiplicativeExpression} , \newline
    \{ ( \lstinline{'+'} @ \lstinline{'-'} ) , {\sffamily MultiplicativeExpression} \};;;
    ShiftExpression :::=
    {\sffamily AdditiveExpression} , \newline
    \{ ( \lstinline{'<<'} @ \lstinline{'>>'} ) , {\sffamily AdditiveExpression} \};;;
    RelationalExpression :::=
    {\sffamily ShiftExpression} , \newline
    \{ ( \lstinline{'<'} @ \lstinline{'>'} @ \lstinline{'<='} @ \lstinline{'>='} ) , {\sffamily ShiftExpression} \};;;
    EqualityExpression :::=
    {\sffamily RelationalExpression} , \newline
    \{ ( \lstinline{'=='} @ \lstinline{'\!='} ) , {\sffamily RelationalExpression} \};;;
    ANDExpression :::=
    {\sffamily EqualityExpression} , \{ \lstinline{'\&'} , {\sffamily EqualityExpression} \};;;
    ExclusiveORExpression :::=
    {\sffamily ANDExpression} , \{ \lstinline{'^'} , {\sffamily ANDExpression} \};;;
    InclusiveORExpression :::=
    {\sffamily ExclusiveORExpression} , \newline
    \{ \lstinline{'|'} , {\sffamily ExclusiveORExpression} \};;;
    LogicalANDExpression :::=
    {\sffamily InclusiveORExpression} , \newline
    \{ \lstinline{'\&\&'} , {\sffamily InclusiveORExpression} \};;;
    LogicalORExpression :::=
    {\sffamily LogicalANDExpression} , \newline
    \{ \lstinline{'||'} , {\sffamily LogicalANDExpression} \};;;
    ConditionalExpression :::=
    {\sffamily LogicalORExpression} , \newline
    [ \lstinline{'\?'} , {\sffamily Expression} , \lstinline{':'} , {\sffamily ConditionalExpression} ];;;
    AssignmentExpression :::=
    ! {\sffamily UnaryExpression} , {\sffamily AssignmentOperator} , {\sffamily AssignmentExpression}
    ! {\sffamily ConditionalExpression}
    ;;;
    AssignmentOperator :::=
    ! \lstinline{'='} // simple assign
    ! \lstinline{'*='}
    @ \lstinline{'\/='}
    @ \lstinline{'\%='}
    @ \lstinline{'+='}
    @ \lstinline{'-='}
    @ \lstinline{'<<='}
    @ \lstinline{'>>='}
    @ \lstinline{'\&='}
    @ \lstinline{'^='}
    @ \lstinline{'|='} // compound assign
    ;;;
    Expression :::=
    {\sffamily AssignmentExpression} , \newline
    \{ \lstinline{','} , {\sffamily AssignmentExpression} \} // comma operator
    ;;;
    \end{bnf}
}

\bigskip
\subsection{Declarations}
\bigskip
\centerline{
    \begin{bnf}[colspec={Q[0mm]Q[r,font=\sffamily]Q[r]Q[l,mode=text,font=\ttfamily,98mm]Q[l]}]
    Declaration :::=
    {\sffamily DeclarationSpecifiers} , [ {\sffamily InitDeclaratorList} ] , \lstinline{';'}
    ;;;
    DeclarationSpecifiers :::=
    ( {\sffamily StorageClassSpecifier} @ {\sffamily TypeSpecifier} @ {\sffamily TypedefName} ) , \newline
    \{ {\sffamily StorageClassSpecifier} @ {\sffamily TypeSpecifier} @ {\sffamily TypedefName} \}
    // specifiers
    ;;;
    InitDeclaratorList :::=
    {\sffamily InitDeclarator} , \{ \lstinline{','} , {\sffamily InitDeclarator} \};;;
    InitDeclarator :::=
    {\sffamily Declarator} , [ \lstinline{'='} , {\sffamily Initializer} ];;;
    StorageClassSpecifier :::=
    \lstinline{'typedef'}
    ;;;
    TypeSpecifier :::=
    ! \lstinline{'void'}
    @ \lstinline{'char'}
    @ \lstinline{'short'}
    @ \lstinline{'int'}
    @ \lstinline{'long'}
    @ \lstinline{'signed'}
    ! \lstinline{'unsigned'}
    @ \lstinline{'\_Bool'}
    @ {\sffamily StructOrUnionSpecifier}
    ;;;
    StructOrUnionSpecifier :::=
    ! \lstinline{'struct'} , [ {\sffamily Identifier} ] , \newline
    \lstinline{'\{'} , {\sffamily StructDeclarationList} , \lstinline{'\}'}
    ! \lstinline{'struct'} , {\sffamily Identifier}
    ;;;
    StructDeclarationList :::=
    {\sffamily StructDeclaration} , \{ {\sffamily StructDeclaration} \};;;
    StructDeclaration :::=
    {\sffamily SpecifierQualifierList} , [ {\sffamily StructDeclaratorList} ] , \lstinline{';'}
    ;;;
    SpecifierQualifierList :::=
    ( {\sffamily TypeSpecifier} @ {\sffamily TypedefName} ) , \newline
    \{ {\sffamily TypeSpecifier} @ {\sffamily TypedefName} \}
    ;;;
    StructDeclaratorList :::=
    {\sffamily StructDeclarator} , \{ \lstinline{','} , {\sffamily StructDeclarator} \};;;
    StructDeclarator :::=
    {\sffamily Declarator}
    ;;;
    \end{bnf}
}

\centerline{
    \begin{bnf}[colspec={Q[0mm]Q[r,font=\sffamily]Q[r]Q[l,mode=text,font=\ttfamily,90mm]Q[l]}]
    Declarator :::=
    [ {\sffamily Pointer} ] , {\sffamily DirectDeclarator};;;
    DirectDeclarator :::=
    ( {\sffamily Identifier} @  ( \lstinline{'('} , {\sffamily Declarator} , \lstinline{')'} )  ) , \newline
    \{ {\sffamily DirectDeclaratorPart} \}
    ;;;
    DirectDeclaratorPart :::=
    ! \lstinline{'['} , [ {\sffamily IntegerConstant} ] , \lstinline{']'}
    // array declaration
    ! \lstinline{'('} , [ {\sffamily ParameterList} ] , \lstinline{')'}
    // function declaration
    ;;;
    Pointer :::=
    \lstinline{'*'} , \{ \lstinline{'*'} \};;;
    ParameterList :::=
    {\sffamily ParameterDeclaration} , \newline
    \{ \lstinline{','} , {\sffamily ParameterDeclaration} \};;;
    ParameterDeclaration :::=
    {\sffamily DeclarationSpecifiers} , \newline
    [ {\sffamily Declarator} @ {\sffamily AbstractDeclarator} ];;;
    TypeName :::=
    {\sffamily SpecifierQualifierList} , [ {\sffamily AbstractDeclarator} ];;;
    AbstractDeclarator :::=
    ! [ {\sffamily Pointer} ] , {\sffamily DirectAbstractDeclarator}
    ! {\sffamily Pointer}
    ;;;
    DirectAbstractDeclarator :::=
    ! \lstinline{'('} , {\sffamily AbstractDeclarator} , \lstinline{')'}
    ! [ \lstinline{'('} , {\sffamily AbstractDeclarator} , \lstinline{')'} ] , \newline
    ( \lstinline{'['} , [ {\sffamily IntegerConstant} ] , \lstinline{']'} @ \newline
    \lstinline{'('} , [ {\sffamily ParameterList} ] , \lstinline{')'} ) , \newline
    \{ \lstinline{'['} , [ {\sffamily IntegerConstant} ] , \lstinline{']'} @ \newline
    \lstinline{'('} , [ {\sffamily ParameterList} ] , \lstinline{')'} \}
    ;;;
    TypedefName :::=
    {\sffamily Identifier};;;
    Initializer :::=
    ! {\sffamily AssignmentExpression}
    ! \lstinline{'\{'} , {\sffamily InitializerList} , [ \lstinline{','} ] , \lstinline{'\}'}
    ;;;
    InitializerList :::=
    [ {\sffamily Designation} ] , {\sffamily Initializer} , \newline
    \{ \lstinline{','} , [ {\sffamily Designation} ] , {\sffamily Initializer} \};;;
    Designation :::=
    {\sffamily DesignatorList} , \lstinline{'='};;;
    DesignatorList :::=
    {\sffamily Designator} , \{ {\sffamily Designator} \};;;
    Designator :::=
    ! \lstinline{'['} , {\sffamily IntegerConstant} , \lstinline{']'}
    ! \lstinline{'.'} , {\sffamily Identifier}
    \end{bnf}
}

\bigskip
\subsection{Statements}
\medskip
\centerline{
    \begin{bnf}[colspec={Q[0mm]Q[r,font=\sffamily]Q[r]Q[l,mode=text,font=\ttfamily,90mm]Q[l]}]
    Statement :::=
    ! {\sffamily CompoundStatement}
    ! {\sffamily ExpressionStatement}
    ! {\sffamily SelectionStatement}
    ! {\sffamily IterationStatement}
    ! {\sffamily JumpStatement}
    ;;;
    CompoundStatement :::=
    \lstinline{'\{'} , [ {\sffamily BlockItemList} ] , \lstinline{'\}'};;;
    BlockItemList :::=
    {\sffamily BlockItem} , \{ {\sffamily BlockItem} \};;;
    BlockItem :::=
    ! {\sffamily Statement}
    ! {\sffamily Declaration}
    ;;;
    ExpressionStatement :::=
    [ {\sffamily Expression} ] , \lstinline{';'};;;
    SelectionStatement :::=
    \lstinline{'if'} , \lstinline{'('} , {\sffamily Expression} , \lstinline{')'} , {\sffamily Statement} , \newline
    [ \lstinline{'else'} , {\sffamily Statement} ]
    ;;;
    IterationStatement :::=
    ! \lstinline{'while'} , \lstinline{'('} , {\sffamily Expression} , \lstinline{')'} , {\sffamily Statement}
    ! \lstinline{'do'} , {\sffamily Statement} , \newline
    \lstinline{'while'} , \lstinline{'('} , {\sffamily Expression} , \lstinline{')'} , \lstinline{';'}
    ! 
    \lstinline{'for'} , \lstinline{'('} , [ {\sffamily Expression} ] , \lstinline{';'} , \newline
    \makebox[71pt]{} [ {\sffamily Expression} ] , \lstinline{';'} , \newline
    \makebox[71pt]{} [ {\sffamily Expression} ] , \lstinline{')'} , {\sffamily Statement}
    \end{bnf}
}

\centerline{
    \begin{bnf}[colspec={Q[0mm]Q[r,font=\sffamily]Q[r]Q[l,mode=text,font=\ttfamily,90mm]Q[l]}]
    JumpStatement :::=
    ! \lstinline{'continue'} , \lstinline{';'}
    ! \lstinline{'break'} , \lstinline{';'}
    ! \lstinline{'return'} , [ {\sffamily Expression} ] , \lstinline{';'}
    \end{bnf}
}
    
\bigskip
\subsection{External Definitions}
\bigskip
\centerline{
    \begin{bnf}[colspec={Q[0mm]Q[r,font=\sffamily]Q[r]Q[l,mode=text,font=\ttfamily,90mm]Q[l]}]
    TranslationUnit :::=
    {\sffamily ExternalDeclaration} , \{ {\sffamily ExternalDeclaration} \}
    // \hyperref[sec:extdef]{external definitions}
    ;;;
    ExternalDeclaration :::=
    ! {\sffamily FunctionDefinition}
    ! {\sffamily Declaration}
    ;;;
    FunctionDefinition :::=
    {\sffamily DeclarationSpecifiers} , {\sffamily Declarator} , {\sffamily CompoundStatement}
    // \hyperref[subsec:fndef]{function definitions}
    ;;;
    DeclarationList :::=
    {\sffamily Declaration} , \{ {\sffamily Declaration} \}
    \end{bnf}
}


\end{document}