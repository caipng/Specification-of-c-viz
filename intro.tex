\newpage
\section*{Introduction}\label{sec:intro}
\addcontentsline{toc}{section}{Introduction}
\markright{INTRODUCTION}

This document follows the structure of (and frequently refers to) the ISO/IEC
9899:1999 - n1256 version of the C99 standard\footnote{\href{https://www.open-std.org/jtc1/sc22/wg14/www/docs/n1256.pdf}{https://www.open-std.org/jtc1/sc22/wg14/www/docs/n1256.pdf}}.
Where applicable, we express grammars using Extended Backus–Naur form\footnote{We use $\Coloneqq$ for definition, \lstinline{,} for concatenation, \lstinline{|} for alternation, $[\ x\ ]$ for optional $x$, $\{\ x\ \}$ for zero or more repetitions of $x$, $(\ \dots\ )$ for grouping, {\textsf `}$\dots${\textsf '} for terminals, $-$ for negation and {\textsf ?}$\dots${\textsf ?} for special sequences.}, adopting the ISO/IEC 14977 standard proposed by R. S. Scowen.
\par
\deviation{
Deviations from the \Verb!C! standard will be on their own paragraph and highlighted like this.
}
\vspace{10pt}
\par
\Secref{sec:concepts} describes various language concepts, such as scopes of identifiers and types.
\Secref{sec:lexgrammar} to \secref{sec:extdef} detail the syntax, type checking constraints and semantics of various language constructs.
In \secref{sec:memory} we look at the memory related runtime data structures implemented in the interpreter.
Lastly, an informal description of the big step operational semantics of the language is provided in \secref{sec:opsem}.
