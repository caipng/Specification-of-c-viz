\newpage
\section{Concepts}\label{sec:concepts}

\subsection{Numerical limits}\label{subsec:numlimits}

The following are the constant values defined for the numerical limits of integer types.
A future version of \Verb|c-viz| may allow for configuration of these values.

\begin{table}[h]
\begin{tabular}{>{\ttfamily \bfseries}r>{$}l<{$}l}
CHAR\_BIT    &  8 &  number of bits for smallest object (byte)\\
SCHAR\_MIN   &  -(2^7-1) &  minimum value for an object of type \Verb!signed char!\\
SCHAR\_MAX   &  2^7 - 1 &  maximum value for an object of type \Verb!signed char!\\
UCHAR\_MAX   &  2^8 - 1 &  maximum value for an object of type \Verb!unsigned char!\\
CHAR\_MIN    &  -(2^7-1) &  minimum value for an object of type \Verb!char!\\
CHAR\_MAX    &  2^7 - 1 &  maximum value for an object of type \Verb!char!\\
MB\_LEN\_MAX &  1 & maximum number of bytes in a multibyte character\\
SHRT\_MIN    &  -(2^{15}-1)&  minimum value for an object of type \Verb!short int!\\
SHRT\_MAX    &  2^{15}-1&  maximum value for an object of type \Verb!short int!\\
USHRT\_MAX   &  2^{16}-1&  maximum value for an object of type \Verb!unsigned short int!\\
INT\_MIN     &  -(2^{31}-1)&  minimum value for an object of type \Verb!int!\\
INT\_MAX     &  2^{31}-1&  maximum value for an object of type \Verb!int!\\
UINT\_MAX    &  2^{32}-1&  maximum value for an object of type \Verb!unsigned int!\\
LONG\_MIN    &  -(2^{31}-1)&  minimum value for an object of type \Verb!long int!\\
LONG\_MAX    &  2^{31}-1&  maximum value for an object of type \Verb!long int!\\
ULONG\_MAX   &  2^{32}-1&  maximum value for an object of type \Verb!unsigned long int!\\
LLONG\_MIN   &  -(2^{63}-1)&  minimum value for an object of type \Verb!long long int!\\
LLONG\_MAX   &  2^{63}-1&  maximum value for an object of type \Verb!long long int!\\
ULLONG\_MAX  &  2^{64}-1& maximum value for an object of type \Verb!unsigned long long int!
\end{tabular}
\end{table}

\subsection{Scopes of identifiers}\label{subsec:scopes}
\spec{
An identifier can denote an object; a function; a tag or a member of a structure; or a typedef name. The same identifier can denote different entities at different points in the program.
\par
For each different entity designated by an identifier, the identifier can only be used within a region of program code called its \emph{scope}. Thus an identifier can designate different entities only if they are in different scopes or in different name spaces.
\par
\deviation{
There are 2 kinds of scopes: file and block. Function prototype scope and function scope\tablefootnote{used only by label names for \Verb!goto! constructs} are not supported.
}
\par

}

\subsection{Types}\label{subsec:types}