\newpage
\section{Concepts}\label{sec:concepts}

\subsection{Numerical limits}\label{subsec:numlimits}

The following are the constant values defined for the numerical limits of integer types.
A future version of \Verb|c-viz| may allow for configuration of these values.

\begin{table}[h]
\begin{tabular}{>{\ttfamily \bfseries}r>{$}l<{$}l}
CHAR\_BIT    &  8 &  number of bits for smallest object (byte)\\
SCHAR\_MIN   &  -(2^7-1) &  minimum value for an object of type \Verb!signed char!\\
SCHAR\_MAX   &  2^7 - 1 &  maximum value for an object of type \Verb!signed char!\\
UCHAR\_MAX   &  2^8 - 1 &  maximum value for an object of type \Verb!unsigned char!\\
CHAR\_MIN    &  -(2^7-1) &  minimum value for an object of type \Verb!char!\\
CHAR\_MAX    &  2^7 - 1 &  maximum value for an object of type \Verb!char!\\
MB\_LEN\_MAX &  1 & maximum number of bytes in a multibyte character\\
SHRT\_MIN    &  -(2^{15}-1)&  minimum value for an object of type \Verb!short int!\\
SHRT\_MAX    &  2^{15}-1&  maximum value for an object of type \Verb!short int!\\
USHRT\_MAX   &  2^{16}-1&  maximum value for an object of type \Verb!unsigned short int!\\
INT\_MIN     &  -(2^{31}-1)&  minimum value for an object of type \Verb!int!\\
INT\_MAX     &  2^{31}-1&  maximum value for an object of type \Verb!int!\\
UINT\_MAX    &  2^{32}-1&  maximum value for an object of type \Verb!unsigned int!\\
LONG\_MIN    &  -(2^{31}-1)&  minimum value for an object of type \Verb!long int!\\
LONG\_MAX    &  2^{31}-1&  maximum value for an object of type \Verb!long int!\\
ULONG\_MAX   &  2^{32}-1&  maximum value for an object of type \Verb!unsigned long int!\\
LLONG\_MIN   &  -(2^{63}-1)&  minimum value for an object of type \Verb!long long int!\\
LLONG\_MAX   &  2^{63}-1&  maximum value for an object of type \Verb!long long int!\\
ULLONG\_MAX  &  2^{64}-1& maximum value for an object of type \Verb!unsigned long long int!
\end{tabular}
\end{table}

\subsection{Scopes of identifiers}\label{subsec:scopes}
\spec{
An identifier can denote an object; a function; a tag or a member of a structure; or a typedef name. The same identifier can denote different entities at different points in the program.
\par
The \emph{scope} of an identifier is a region of program code in which the identifier may be used to refer to a particular entity.
An identifier can designate different entities only if they are in different scopes or in different name spaces.
\par
\deviation{
There are two kinds of scopes: file and block. Function prototype scope and function scope\tablefootnote{used only by label names for \Verb!goto! constructs} are not supported.
}
\par
The scope of an identifier is determined by where its declaration appears.
If the declaration appears outside of any block, the identifier has \emph{file scope}, which terminates at the end of the translation unit.
If it appears inside a block, the identifier has \emph{block scope}, which terminates at the end of the block.
\par
If two different entities existing in the same name space have the same identifier, then the scope of one of them (the \emph{inner scope}) must be a strict subset of the other (the \emph{outer scope}). 
Within the inner scope, the identifier will designate the entity declared within and the entity declared in the outer scope is \emph{hidden}.
Two identifiers is said to have the \emph{same scope} if their scopes terminate at the same point.
\par
Structure tags have scopes beginning just after the appearance of the tag in the type specifier portion of its declaration.
All other identifiers have scope beginning just after the completion of its declarator.
}

\subsection{Name spaces of identifiers}\label{subsec:namespaces}
\spec{
If more than one declaration of an identifier is visible at any point, the context in which the identifier is used disambiguates what entity it refers to.
Thus, there are separate \emph{name spaces} for identifiers:
\begin{itemize}
    \item the \emph{tags} of structures (appears right after the keyword \Verb!struct!)
    \item the \emph{members} of structures; each structure has a separate name space for its members (disambiguated by the type of the expression used to access the member)
    \item all other identifiers
\end{itemize}
}

\subsection{Storage duration of objects}\label{subsec:storageduration}
\spec{
An object has a \emph{storage duration} that determines its lifetime.
There are three storage durations: static, automatic, and allocated.
\par
The \emph{lifetime} of an object is the portion of program execution during which memory is allocated to it.
Hence, an object has a constant address and retains its last-stored value throughout its lifetime.
\par
\deviation{
If an object is referred to outside of its lifetime, the interpreter throws the error\\
\Verb!"no object allocated at ..."!. Future versions of \Verb!c-viz! should allow for such runtime checks to be disabled.
}
\par
\deviation{Pointers retain their values when objects they point to reaches the end of their lifetime.
}
\par
An object designated by an identifier with file scope has \emph{static storage duration}, with its lifetime spanning the entire execution of the program\footnote{program execution begins on entry to the \Verb!main! function and ends when the \Verb!main! function returns}.
\par
An object designated by an identifier with block scope has \emph{automatic storage duration}.
Its lifetime extends from entry into the block until execution of that block ends\footnote{the corresponding \Verb!EXIT\_BLOCK! instruction is popped from the control stack}.
The initial value of the object is indeterminate.
\par
\emph{Allocated storage duration} is described in \subsecref{subsec:memmgtfns}.
}

\subsection{Types}\label{subsec:types}
\spec{
Types give meaning to values stored in objects. They are partitioned into:
\begin{itemize}
    \item \emph{object types} which fully describe objects
    \item \emph{function types} that describe functions
    \item \emph{incomplete types} that describe objects (or \Verb!void!) but lack information to determine their sizes
\end{itemize}
\par
The types supported by \Verb!c-viz! are: \Verb!\_Bool!, \Verb!char!, \Verb!signed char!, \Verb!unsigned char!, \Verb!short int!,\\ \Verb!unsigned short int!, \Verb!int!, \Verb!unsigned int!, \Verb!long int!, \Verb!unsigned long int!, \Verb!long long int!,\\ \Verb!unsigned long long int!, \Verb!void!, arrays, structures, functions and pointers.
Other than functions (function type) and \Verb!void! (incomplete type), the other types are object types.
\par
The \emph{signed integer types} consists of \Verb!signed char!, \Verb!short int!, \Verb!int!, \Verb!long int! and \Verb!long long int!.
\par
For each of the signed integer types, there is a corresponding unsigned integer type designated with the keyword \Verb!unsigned!.
They use the same amount of storage and have the same alignment requirements.
These types, combined with the type \Verb!\_Bool!, make up the \emph{unsigned integer types}.
\par
If a signed integer type can represent a range of nonnegative values $S$ and its corresponding unsigned integer type can represent a range of values $U$, then $S \subseteq U$.
Furthermore, the representation of any $x \in S$ is the same for both types.
\par
Computations involving unsigned operands can never overflow.
A result $x$ that cannot be represented by the resulting unsigned integer type with maximum value $M$ will be reduced to \\$x \mod (M + 1)$.
\par
\deviation{The interpreter throws an error on signed integer overflows.}
\par
\deviation{
Real and complex floating types are currently not supported.
}
\par
The type \Verb!char! along with the signed and unsigned integer types make up the \emph{integer types} and \emph{arithmetic types}.
The three types \Verb!char!, \Verb!signed char! and \Verb!unsigned char! make up the \emph{character types}.
\par
\deviation{
The type \Verb!char! is defined to have the same range, representation and behaviour as \Verb!signed char!.
}
\par
The \Verb!void! type denotes an empty set of values and is an incomplete type that cannot be completed.
\par
Any number of \emph{derived types} may be constructed from the object, function and incomplete types:
\begin{itemize}
    \item An \emph{array} type describes a contiguously allocated nonempty set of objects with a particular \emph{element type}.
    Arrays are characterized by their element type and the number of elements.
    \item A \emph{structure} type describes a sequentially allocated nonempty\footnote{empty structures are supported by some \Verb!C! compilers and is also a valid \Verb!C++! construct} set of member objects, each of which has an optionally specified name.
    \item A \emph{function} type describes a function with some return type.
    It is characterized by its return type and the number and types of its parameters.
    \item A \emph{pointer} type may be derived from any type, called the \emph{referenced type}.
    It describes an object whose value is the memory address of an entity of the referenced type.
\end{itemize}
These derived types may be constructed recursively.
\par
\deviation{
\emph{Union} types are currently not supported.
}
\par
The \emph{scalar types} consists of arithmetic and pointer types.
The \emph{aggregate types} consist of array and structure types.
\par
\deviation{Array types of unknown sizes\tablefootnote{arrays of unknown sizes \Verb!T[]! in function parameter lists can be replaced by corresponding pointer types instead; their only other use in \Verb!C! is within \Verb!extern! declarations, which are not supported} or variable length are not supported.}
\par
A structure type of unknown content is an incomplete type.
It can be completed by declaring the same structure tag with its defining content later in the same scope\footnote{see example in \secref{sec:extdef}}.
\par
A type has known constant size if the type is not incomplete.
\par
\deviation{
Only \emph{unqualified} types are supported.
}
\par
\deviation{
Pointers to any type have the same representation and alignment requirements.
}
}

\subsubsection{Size and alignment}\label{subsubsec:sizealign}
\spec{
Every complete object type has two properties, \emph{alignment requirement} and \emph{size}.
The size of an object is the number of bytes in memory that it occupies.
The alignment requirement of an object is a non-negative integral power of two representing the number of bytes between successive addresses at which objects of this type can be allocated.
\par
The weakest (smallest) alignment is the alignment of the types \Verb!char!, \Verb!signed char! and \\ \Verb!unsigned char!, which equals 1.
\par
An object of type \Verb!int! has large enough size to contain any value in the range \textbf{\texttt{INT\_MIN}} to \textbf{\texttt{INT\_MAX}} as defined in \subsecref{subsec:numlimits}.
\par
The values for the alignment requirements and sizes of the scalar types used by \Verb!c-viz! are typical for 32-bit x86 architectures.
Future versions of \Verb!c-viz! may allow configuration of these values.
\begin{table}[h]
\centering
\begin{tabular}[t]{>{\ttfamily \bfseries}r>{\ttfamily $}l<{$}}
CHAR\_SIZE   & 1 \\
SCHAR\_SIZE  & \text{CHAR\_SIZE} \\
UCHAR\_SIZE  & \text{CHAR\_SIZE} \\
SHRT\_SIZE   & 2 \\
USHRT\_SIZE  & \text{SHRT\_SIZE} \\
INT\_SIZE    & 4 \\
UINT\_SIZE   & \text{INT\_SIZE} \\
LONG\_SIZE   & 4 \\
ULONG\_SIZE  & \text{LONG\_SIZE} \\
LLONG\_SIZE  & 8 \\
ULLONG\_SIZE & \text{LLONG\_SIZE} \\
PTR\_SIZE    & \text{INT\_SIZE} 
\end{tabular}
\begin{tabular}[t]{>{\ttfamily \bfseries}r>{\ttfamily $}l<{$}}
CHAR\_ALIGN   & 2^0 \\
SCHAR\_ALIGN  & 2^0 \\
UCHAR\_ALIGN  & \text{CHAR\_ALIGN} \\
SHRT\_ALIGN   & 2^1 \\
USHRT\_ALIGN  & \text{SHRT\_ALIGN} \\
INT\_ALIGN    & 2^2 \\
UINT\_ALIGN   & \text{INT\_ALIGN} \\
LONG\_ALIGN   & 2^2 \\
ULONG\_ALIGN  & \text{LONG\_ALIGN} \\
LLONG\_ALIGN  & 2^3 \\
ULLONG\_ALIGN & \text{LLONG\_ALIGN} \\
PTR\_ALIGN    & \text{INT\_ALIGN} \\
MAX\_ALIGN    & \text{LLONG\_ALIGN}
\end{tabular}
\end{table}
}

\subsubsection{Representation}\label{subsubsec:typerep}
\spec{
Values stored in objects consist of $n\ \times\ $\textbf{\texttt{CHAR\_BIT}} bits, where $n$ is the size of the object in bytes.
The value may be copied into an object of type \Verb!unsigned char [n]! and the resulting set of bytes is called the \emph{object representation} of the value.
Two values with the same object representation compare equal, but the converse does not necessarily hold.
\par
Padding bytes in structure types take on unspecified values.
\par
\deviation{
Signed integer types use two's complement representation for negative values\tablefootnote{this is a consequence of the underlying implementation using JavaScript's \Verb!BigInt! type, which ``act as two's complement binary strings" for binary operations}.
}
}

\subsubsection{Compatible types}\label{subsubsec:compatibletypes}
\spec{
Two types have \emph{compatible type} if any of the following hold:
\begin{itemize}
    \item the types are the same
    \item both are structure types such that 
    \begin{itemize}
        \item both either have no tag or the same tag
        \item there is a one-to-one correspondence between their members such that each pair of members are declared with compatible types and the same name
        \item the corresponding members shall be declared in the same order
    \end{itemize}
    \item both are array types such that their element types are compatible and their lengths are equal
    \item both are pointer types such that their referenced types are compatible
    \item both are function types such that 
    \begin{itemize}
        \item both specify compatible return types
        \item both have the same arity
        \item corresponding parameters shall have compatible types
    \end{itemize}
\end{itemize}
}

\subsection{Conversions}\label{subsec:conversions}
\spec{
Where an expression is used in a context where a value of different type is expected, \emph{conversion} may occur.
\par
\begin{lstlisting}^^J
int n = 1L; // expression 1L has type long, but int is expected^^J
char *p = malloc(10); // expression malloc(10) has type void*, but char* is expected^^J
\end{lstlisting}
}

\subsubsection{Conversions as if by assignment}
\spec{
Provided that the \hyperref[subsubsec:simpleassgn]{simple assignment} constraints hold, the following conversions happen:
\begin{itemize}
    \item In the \hyperref[subsec:assgn]{assignment operator}, the value of the right hand operand is converted to the type of the left hand operand.
    \item When \hyperref[subsec:init]{initializing} an object of scalar type, the value of the initializer expression is converted to the type of the object being initialized.
    \item In a \hyperref[subsubsec:fncall]{function call} expression, the value of each argument expression is converted to the type of the corresponding parameter.
    \item In a \hyperref[subsubsec:return]{return} statement, the value of the \Verb!return! expression is converted to an object having the return type of the function.
\end{itemize}
}

\subsubsection{Usual arithmetic conversions}
\spec{
The operands of the following operators undergo implicit conversions to obtain a common type, in which the calculation is performed:
\begin{itemize}
    \item \hyperref[subsec:mult]{multiplicative operators}: \Verb!*, /, \%!
    \item \hyperref[subsec:add]{additive operators}: \Verb!+, -!
    \item \hyperref[subsec:rel]{relational operators}: \Verb!<, >, <=, >=!
    \item \hyperref[subsec:eq]{equality operators}: \Verb|==, !=|
    \item \hyperref[subsec:bitand]{bitwise AND operator}: \Verb!\&!
    \item \hyperref[subsec:bitxor]{bitwise exclusive OR operator}: \string^
    \item \hyperref[subsec:bitor]{bitwise inclusive OR operator}: \Verb!|!
    \item \hyperref[subsec:cond]{conditional operator}: \Verb!?:!
\end{itemize}
\par
The conversion proceeds as follows:
\begin{itemize}
    \item Both operands undergo \hyperref[subsubsubsec:intpromo]{integer promotions}. 
    \item If the types are the same, we are done.
    \item Else the types are different:
    \begin{itemize}
        \item If the types have the same signedness, the operand whose type has the smaller \emph{conversion rank}\footnote{see \subsubsecref{subsubsubsec:intpromo}} is implicitly converted\footnote{see \subsubsecref{subsubsubsec:intconv}} to the other type.
        \item Else the operands have different signedness:
        \begin{itemize}
            \item If the unsigned type has \emph{conversion rank} greater than or equal to the rank of the signed type, then the operand with the signed type is implicitly converted to the unsigned type.
            \item Else the unsigned type has \emph{conversion rank} less than the signed type:
            \begin{itemize}
                \item If the signed type can represent all values of the unsigned type, then the operand with the unsigned type is implicitly converted to the signed type.
                \item Else both operands undergo implicit conversion to the unsigned type counterpart of the signed operand's type.
            \end{itemize}
        \end{itemize}
    \end{itemize}
\end{itemize}
}

\subsubsection{Value categories}
\spec{
Every \hyperref[sec:expr]{expression} in C is characterized by two independent properties: a type and a value category.
\par
\deviation{Every expression belongs to one of two value categories: lvalue and rvalue.}
}

\subsubsection*{Lvalue expressions}\label{subsubsubsec:lvalue}
\spec{
A \emph{lvalue} expression is any expression with object type which designates an object.
\par
Lvalue expressions can be used in the following contexts:
\begin{itemize}
    \item as the operand of the \hyperref[subsubsec:addrindirop]{address-of} operator
    \item as the operand of the pre/post increment and decrement operators
    \item as the left-hand operand of the \hyperref[subsubsec:structmem]{member access} (dot) operator
    \item as the left-hand operand of the \hyperref[subsec:assgn]{assignment} operators
\end{itemize}
\par
If a lvalue expression is used in any context other than \Verb!sizeof! or the operators listed above, non-array lvalues undergo \hyperref[subsubsubsec:lvalueconv]{lvalue conversion}, which loads the value of the object from memory.
\par
The following expressions are lvalues:
\begin{itemize}
    \item identifiers that do not designate a function
    \item string literals
    \item parenthesized expressions $(x)$ if the unparanthesized expression $x$ is a lvalue
    \item the result of a member access (dot) operator if the left-hand operand is a lvalue
    \item the result of a member access through pointer \Verb!->! operator
    \item the result of the indirection \Verb!*! operator applied to a pointer to object
    \item the result of the subscript \Verb![]! operator 
\end{itemize}
}

\subsubsection*{Modifiable lvalue expressions}\label{subsubsubsec:modifiablelvalue}
\spec{
A \emph{modifiable lvalue} is any lvalue expression of non-array type.
\par
Only modifiable lvalue expressions may be used as the left-hand operand of assignment operators and as operands of increment/decrement operators.
}

\subsubsection*{Rvalue expressions}\label{subsubsubsec:rvalue}
\spec{
A \emph{rvalue} expression is defined by exclusion: an expression is a \emph{rvalue} if it is not a lvalue. The address of a rvalue expression cannot be taken.
\par
The following expressions are rvalues:
\begin{itemize}
    \item integer and character constants
    \item all operators not specified to return lvalues, such as
    \begin{itemize}
        \item function call expressions
        \item cast expressions
        \item member access (dot) operator applied to a non-lvalue structure
        \item results of all arithmetic, relational, logical and bitwise operators
        \item results of increment/decrement operators
        \item results of assignment operators\footnote{lvalue in \Verb!C++!}
        \item conditional operator
        \item comma operator
        \item address-of operator
    \end{itemize}
\end{itemize}
}

\subsubsection{Value transformations}
\subsubsection*{Lvalue conversion}\label{subsubsubsec:lvalueconv}
\spec{
Any \hyperref[subsubsubsec:lvalue]{lvalue} expression of non-array type when used in any context other than
\begin{itemize}
    \item as the operand of the address-of operator
    \item as the operand of the pre/post increment and decrement operators
    \item as the left-hand operand of the member access (dot) operator
    \item as the left-hand operand of the assignment operators
    \item as the operand of \Verb!sizeof!
\end{itemize}
undergoes \emph{lvalue conversion}: the type and value remains the same but it loses its lvalue properties. Specifically, the address may no longer be taken.
\par
This conversion models the memory load of the object.
\par
\begin{lstlisting}^^J
int x = n; // lvalue conversion on n, loads object from memory^^J
int *p = &n; // no lvalue conversion, does not read from memory^^J
\end{lstlisting}
}

\subsubsection*{Array to pointer conversion}
\spec{
Any \hyperref[subsubsubsec:lvalue]{lvalue} expression of array type when used in any context other than
\begin{itemize}
    \item as the operand of the address-of operator
    \item as the operand of the \Verb!sizeof!
\end{itemize}
undergoes a conversion to the non-lvalue pointer to its first element.
\par
\begin{lstlisting}^^J
int a[3], b[3][4];^^J
int *p = a; // conversion to &a[0]^^J
int (*q)[4] = b; // conversion to &b[0]^^J
\end{lstlisting}
}

\subsubsection*{Function to pointer conversion}
\spec{
Any function designator expression when used in any context other than
\begin{itemize}
    \item as the operand of the address-of operator
    \item as the operand of the \Verb!sizeof!
\end{itemize}
undergoes a conversion to the non-lvalue pointer to the function designated by the expression.
\par
\begin{lstlisting}^^J
void f() \{ return; \}^^J
void (*p)() = f; // conversion to &f^^J
(***p)(); // repeated conversion to &f and dereference to f^^J
\end{lstlisting}
}

\subsubsection{Implicit conversions}
\spec{
\emph{Implicit conversion}, whether as if by assignment or a usual arithmetic conversion, consists of:
\begin{enumerate}
    \item value transformation, if applicable
    \item one of the conversions listed below, if it produces the target type
\end{enumerate}
}

\subsubsection*{Compatible types}
\spec{
Conversion of a value of any type to any \hyperref[subsubsec:compatibletypes]{compatible type} is a no-op.
}

\subsubsection*{Integer promotions}\label{subsubsubsec:intpromo}
\spec{
\emph{Integer promotion} is the implicit conversion of a value of any integer type with \emph{rank} less than or equal to \emph{rank} of int to the value of type \Verb!int! or \Verb!unsigned int!.
\par
If \Verb!int! can represent the entire range of values of the original type, the value is converted to type \Verb!int!. Else, the value is converted to \Verb!unsigned int!.
\par
Integer promotions preserve the value, including the sign.
\par
\emph{Rank} is a property defined for all integer types:
\begin{itemize}
    \item The ranks of all signed integer types are different and increases with their precision.
    \item The ranks of all signed integer types equal the ranks of their unsigned counterparts.
    \item The rank of \Verb!char! is equal to the rank of \Verb!signed char! and \Verb!unsigned char!.
    \item The rank of \Verb!\_Bool! is less than the rank of any other standard integer type.
    \item Ranking is transitive.
\end{itemize}
\par
The specific values for ranks used by \Verb!c-viz! are given in the following table:
\begin{table}[h]
\centering
\begin{tabular}[t]{>{\ttfamily}r>{$}l<{$}l}
\toprule
\textbf{\sffamily Integer Type} & \textbf{\sffamily Rank} \\ \midrule
\_Bool                & 0             \\
Char                  & 1             \\
SignedChar            & 1             \\
UnsignedChar          & 1             \\
ShortInt              & 2             \\
UnsignedShortInt      & 2             \\ \bottomrule
\end{tabular}
\hspace{2em}
\begin{tabular}[t]{>{\ttfamily}r>{$}l<{$}l}
\toprule
\textbf{\sffamily Integer Type} & \textbf{\sffamily Rank} \\ \midrule
Int                   & 3             \\
UnsignedInt           & 3             \\
LongInt               & 4             \\
UnsignedLongInt       & 4             \\
LongLongInt           & 5             \\
UnsignedLongLongInt   & 5             \\ \bottomrule
\end{tabular}
\end{table}
\par
Integer promotions are applied in the following contexts:
\begin{itemize}
    \item as part of \emph{usual arithmetic conversions}
    \item to the operand of the unary arithmetic operators \Verb!+! and \Verb!-!
    \item to the operand of the unary bitwise operator $\sim$
    \item to both operands of the shift operators \Verb!<<! and \Verb!>>!
\end{itemize}
}

\subsubsection*{Boolean conversions}
\spec{
A value of any scalar type can be implicitly converted to \Verb!\_Bool!. Values that compare equal to zero are converted to \Verb!0! and all other values are converted to \Verb!1!.
}

\subsubsection*{Integer conversions}\label{subsubsubsec:intconv}
\spec{
A value of any integer type can be implicitly converted to any other integer type. Except where covered by promotions and boolean conversions above, the following rules apply:
\begin{itemize}
    \item if the target type can represent the value, the value is unchanged
    \item if the target type is unsigned, modulo arithmetic is applied such that the result fits in the target type
    \item if the target type is signed, the interpreter throws an error
\end{itemize}
\par
\begin{lstlisting}^^J
char x = 1; // int converted to char, result unchanged^^J
unsigned char n = -123456; // target is unsigned, result is 192^^J
signed char m = 123456; // target is signed, error is thrown^^J
sizeof(int) > -1; // false: operator > applies usual arithmetic conversion^^J
\ \ \ \ \ \ \ \ \ \ \ \ \ \ \ \ \ \ \ \ // target type is unsigned, -1 becomes UINT_MAX^^J
\end{lstlisting}
}

\subsubsection*{Pointer conversions}
\spec{
A pointer to \Verb!void! can be implicitly converted to and from any pointer to object type.
If a pointer to object is converted to a pointer to \Verb!void! and back, its value compares equal to the original pointer.
\begin{lstlisting}^^J
int *p = malloc(10 * sizeof(int)); // malloc returns void*^^J
\end{lstlisting}
\par
The integer constant expression \Verb!0! can be implicitly converted to any pointer type. The result is a null pointer value of its type, guaranteed to compare unequal to any non-null pointer value of that type.
\begin{lstlisting}^^J
int *p = 0;^^J 
\end{lstlisting}
}
