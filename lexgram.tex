\newpage
\section{Lexical Grammar}\label{sec:lexgrammar}
\specsyntax{
Token :::=
{\sffamily Keyword} @ {\sffamily Identifier} @ {\sffamily Constant}
// \hyperref[sec:lexgrammar]{token}
! {\sffamily StringLiteral} @ {\sffamily Punctuator};;;
\_ :::=
\{ {\sffamily WhiteSpace} @ {\sffamily LongComment} @ {\sffamily LineComment} \}
// token separator
;;;
WhiteSpace :::=
\lstinline{' '} @ \lstinline{'\\n'} @ \lstinline{'\\r'} @ \lstinline{'\\t'} @ \lstinline{'\\u000b'} @ \lstinline{'\\f'};;;
}

\specsem{
A \emph{token} is the minimal lexical element of the language.
\par
Each token can be separated by \emph{white-space characters} (space, horizontal tab, new-line, vertical tab, and form-feed), comments (\subsecref{subsec:comments}), or both.
}

\subsection{Keywords}\label{subsec:keywords}
\specsyntax{
Keyword :::=
\lstinline{'auto'} @ \lstinline{'break'} @ \lstinline{'case'} @ \lstinline{'char'} @ \lstinline{'const'} @ \lstinline{'continue'} @ \lstinline{'default'} @ \lstinline{'do'} @ \lstinline{'double'} @ \lstinline{'else'} @ \lstinline{'enum'} @ \lstinline{'extern'} @ \lstinline{'float'} @ \lstinline{'for'} @ \lstinline{'goto'} @ \lstinline{'if'} @ \lstinline{'inline'} @ \lstinline{'int'} @ \lstinline{'long'} @ \lstinline{'register'} @ \lstinline{'restrict'} @ \lstinline{'return'} @ \lstinline{'short'} @ \lstinline{'signed'} @ \lstinline{'sizeof'} @ \lstinline{'static'} @ \lstinline{'struct'} @ \lstinline{'switch'} @ \lstinline{'typedef'} @ \lstinline{'union'} @ \lstinline{'unsigned'} @ \lstinline{'void'} @ \lstinline{'volatile'} @ \lstinline{'while'} @ \lstinline{'\_Alignas'} @ \lstinline{'\_Alignof'} @ \lstinline{'\_Atomic'} @ \lstinline{'\_Bool'} @ \lstinline{'\_Complex'} @ \lstinline{'\_Generic'} @ \lstinline{'\_Imaginary'} @ \lstinline{'\_Noreturn'} @ \lstinline{'\_Static\_assert'} @ \lstinline{'\_Thread\_local'}
// \hyperref[subsec:keywords]{keyword};;;
}

\specsem{
The above tokens (case sensitive) are reserved for use as keywords, and shall not be used otherwise (such as for identifiers).
}

\subsection{Identifiers}\label{subsec:identifiers}
\specsyntax{
Identifier :::=
{\sffamily IdentifierNondigit} , \{ {\sffamily IdentifierNondigit} @ {\sffamily Digit} \}
// \hyperref[subsec:identifiers]{identifier};;;
IdentifierNondigit :::= {\sffamily Nondigit} @ {\sffamily UniversalCharacterName}
;;;
Nondigit :::=
{\it ? lowercase or uppercase alphabet ?} @ \lstinline{'\_'}
;;;
Digit :::=
{\it ? digit from 0 to 9 ?}
;;;
}

\specsem{
An identifier is a sequence of nondigit characters (which includes the underscore \Verb!\_!, the lower and upper case Latin letters, and other characters) and digits, which designates one or more entities as described in \subsecref{subsec:scopes}.
}

\subsection{Universal character names}\label{subsec:univnames}
\specsyntax{
UniversalCharacterName :::=
! \lstinline{'\\u'} , {\sffamily HexQuad}
// \hyperref[subsec:univnames]{universal character name}
! \lstinline{'\\U'} , {\sffamily HexQuad} , {\sffamily HexQuad}
;;;
HexQuad :::=
{\sffamily HexadecimalDigit} , {\sffamily HexadecimalDigit} , {\sffamily HexadecimalDigit} , {\sffamily HexadecimalDigit};;;
}

\specsem{
Universal character names are used in identifiers, character constants and string literals to designate characters that are not in the basic character set.
}

\subsection{Constants}\label{subsec:constants}
\specsyntax{
Constant :::=
{\sffamily IntegerConstant} @ {\sffamily CharacterConstant}
// \hyperref[subsec:constants]{constant}
;;;
}

\specsem{
Each constant has a type, determined by its form and value as described later.
}

\subsubsection{Integer constants}\label{subsubsec:intconstants}
\specsyntax{
IntegerConstant :::=
( {\sffamily DecimalConstant} @ {\sffamily HexadecimalConstant} @ {\sffamily OctalConstant} ) , [ {\sffamily IntegerSuffix} ]
// \hyperref[subsubsec:intconstants]{integer constant}
;;;
DecimalConstant :::=
{\sffamily NonzeroDigit} , \{ {\sffamily Digit} \};;;
OctalConstant :::=
\lstinline{'0'} , \{ {\sffamily OctalDigit} \};;;
HexadecimalConstant :::=
{\sffamily HexadecimalPrefix} , {\sffamily HexadecimalDigit} , \newline
\{ {\sffamily HexadecimalDigit} \};;;
HexadecimalPrefix :::=
\lstinline{'0x'} @ \lstinline{'0X'}
;;;
NonzeroDigit :::=
{\it ? digit from 1 to 9 ?}
;;;
OctalDigit :::=
{\it ? digit from 0 to 7 ?}
;;;
HexadecimalDigit :::=
! {\it ? digit from 0 to 9 ?}
! {\it ? lowercase or uppercase alphabet from a to f ?}
;;;
IntegerSuffix :::=
! {\sffamily UnsignedSuffix} , [ {\sffamily LongLongSuffix} @ {\sffamily LongSuffix} ]
! ( {\sffamily LongLongSuffix} @ {\sffamily LongSuffix} ) , [ {\sffamily UnsignedSuffix} ]
;;;
UnsignedSuffix :::=
\lstinline{'u'} @ \lstinline{'U'};;;
LongSuffix :::=
\lstinline{'l'} @ \lstinline{'L'};;;
LongLongSuffix :::=
\lstinline{'ll'} @ \lstinline{'LL'}
}

\specsem{
An integer constant may have a prefix specifying its base and a suffix specifying its type.
\par
The type of an integer constant is the first of the corresponding list in which its value can be represented:
\begin{table}[]
\centering
\begin{tabular}[t]{@{}l>{\ttfamily}l>{\ttfamily}l@{}}
\toprule
\textbf{Suffix}          & \textbf{\sffamily Decimal Constant}                                                                         & \textbf{\sffamily Octal or Hexadecimal Constant}                                                                                             \\ \midrule
none                     & \begin{tabular}[c]{@{}l@{}}int\\ long int\\ long long int\end{tabular}                            & \begin{tabular}[c]{@{}l@{}}int\\ unsigned int\\ long int\\ unsigned long int\\ long long int\\ unsigned long long int\end{tabular} \\ \midrule
u or U                   & \begin{tabular}[c]{@{}l@{}}unsigned int\\ unsigned long int\\ unsigned long long int\end{tabular} & \begin{tabular}[c]{@{}l@{}}unsigned int\\ unsigned long int\\ unsigned long long int\end{tabular}                                  \\ \midrule
l or L                   & \begin{tabular}[c]{@{}l@{}}long int\\ long long int\end{tabular}                                  & \begin{tabular}[c]{@{}l@{}}long int\\ unsigned long int\\ long long int\\ unsigned long long int\end{tabular}                      \\ \midrule
Both u or U and l or L   & \begin{tabular}[c]{@{}l@{}}unsigned long int\\ unsigned long long int\end{tabular}                & \begin{tabular}[c]{@{}l@{}}unsigned long int\\ unsigned long long int\end{tabular}                                                 \\ \midrule
ll or LL                 & long long int                                                                                     & \begin{tabular}[c]{@{}l@{}}long long int\\ unsigned long long int\end{tabular}                                                     \\ \midrule
Both u or U and ll or LL & unsigned long long int                                                                            & unsigned long long int                                                                                                             \\ \bottomrule
\end{tabular}
\end{table}
}

\subsubsection{Character constants}\label{subsubsec:charconstants}
\specsyntax{
CharacterConstant :::=
\lstinline{'''} , {\sffamily CChar} , \lstinline{'''}
// \hyperref[subsubsec:charconstants]{character constant}
;;;
CChar :::=
! {\sffamily EscapeSequence}
! {\it ? UTF-16 char ?} - ( \lstinline{'''} @ \lstinline{'\\n'} @ \lstinline{'\\'} )
;;;
EscapeSequence :::=
! {\sffamily SimpleEscapeSequence} @ {\sffamily OctalEscapeSequence} @ {\sffamily HexadecimalEscapeSequence} @ {\sffamily UniversalCharacterName}
;;;
SimpleEscapeSequence :::=
\lstinline{'\\'} , ( \lstinline{'''} @ \lstinline{'"'} @ \lstinline{'\?'} @ \lstinline{'\\'} @ \lstinline{'a'} @ \lstinline{'b'} @ \lstinline{'f'} @ \lstinline{'n'} @ \lstinline{'r'} @ \lstinline{'t'} @ \lstinline{'v'} );;;
OctalEscapeSequence :::=
\lstinline{'\\'} , {\sffamily OctalDigit} , [ {\sffamily OctalDigit} ] , [ {\sffamily OctalDigit} ];;;
HexadecimalEscapeSequence :::=
\lstinline{'\\x'} , {\sffamily HexadecimalDigit} , \{ {\sffamily HexadecimalDigit} \};;;
}

\subsection{String literals}\label{subsec:stringlit}
\specsyntax{
StringLiteral :::=
\lstinline{'"'} , [ {\sffamily SCharSequence} ] , \lstinline{'"'}
// \hyperref[subsec:stringlit]{string literal}
;;;
SCharSequence :::=
{\sffamily SChar} , \{ {\sffamily SChar} \};;;
SChar :::=
! {\sffamily EscapeSequence}
! {\it ? UTF-16 char ?} - ( \lstinline{'"'} @ \lstinline{'\\n'} @ \lstinline{'\\'} )
;;;
}

\subsection{Punctuators}\label{subsec:punctuators}
\specsyntax{
Punctuator :::=
\lstinline{'['} @ \lstinline{']'} @ \lstinline{'('} @ \lstinline{')'} @ \lstinline{'\{'} @ \lstinline{'\}'} @ \lstinline{'.'} @ \lstinline{'->'} @ \lstinline{'++'} @ \lstinline{'--'} @ \lstinline{'\&'} @ \lstinline{'*'} @ \lstinline{'+'} @ \lstinline{'-'} @ \lstinline{'~'} @ \lstinline{'\!'} @ \lstinline{'\/'} @ \lstinline{'\%'} @ \lstinline{'<<'} @ \lstinline{'>>'} @ \lstinline{'<'} @ \lstinline{'>'} @ \lstinline{'<='} @ \lstinline{'>='} @ \lstinline{'=='} @ \lstinline{'\!='} @ \lstinline{'^'} @ \lstinline{'|'} @ \lstinline{'\&\&'} @ \lstinline{'||'} @ \lstinline{'\?'} @ \lstinline{':'} @ \lstinline{';'} @ \lstinline{'...'} @ \lstinline{'='} @ \lstinline{'*='} @ \lstinline{'\/='} @ \lstinline{'\%='} @ \lstinline{'+='} @ \lstinline{'-='} @ \lstinline{'<<='} @ \lstinline{'>>='} @ \lstinline{'\&='} @ \lstinline{'^='} @ \lstinline{'|='} @ \lstinline{','} @ \lstinline{'\#'} @ \lstinline{'\#\#'} @ \lstinline{'<:'} @ \lstinline{':>'} @ \lstinline{'<\%'} @ \lstinline{'\%>'} @ \lstinline{'\%:'} @ \lstinline{'\%:\%:'}
// \hyperref[subsec:punctuators]{punctuator}
}

\subsection{Comments}\label{subsec:comments}
\specsyntax{
LongComment :::=
\lstinline{'\/*'} , \{ {\it ? UTF-16 char ?} - \lstinline{'*\/'} \} , \lstinline{'*\/'}
// \hyperref[subsec:comments]{comment};;;
LineComment :::=
\lstinline{'\/\/'} , \{ {\it ? UTF-16 char ?} - \lstinline{'\\n'} \}
// \hyperref[subsec:comments]{comment};;;
}

