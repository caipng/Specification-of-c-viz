\newpage
\section{Expressions}\label{sec:expr}

\subsection{Primary expressions}\label{subsec:priexpr}
\specsyntax{
PrimaryExpression :::=
! {\sffamily Identifier}
// \hyperref[subsec:priexpr]{primary expr}
! {\sffamily Constant}
! {\sffamily StringLiteral}
! \lstinline{'('} , {\sffamily Expression} , \lstinline{')'}
// parenthesized expr
;;;
}

\subsection{Postfix operators}\label{subsec:postfixop}
\specsyntax{
PostfixExpression :::=
{\sffamily PrimaryExpression} , \{ {\sffamily PostfixOp} \}
// \hyperref[subsec:postfixop]{postfix operators}
;;;
PostfixOp :::=
! \lstinline{'['} , {\sffamily Expression} , \lstinline{']'} // \hyperref[subsubsec:arraysub]{array subscripting}
! \lstinline{'('} , [ {\sffamily ArgumentExpressionList} ] , \lstinline{')'}  // \hyperref[subsubsec:fncall]{function call}
! \lstinline{'.'} , {\sffamily Identifier} // \hyperref[subsubsec:structmem]{structure member}
! \lstinline{'->'} , {\sffamily Identifier} // \hyperref[subsubsec:structmem]{structure member}
! \lstinline{'++'} 
@ \lstinline{'--'} // \hyperref[subsubsec:postfixincrdecr]{postfix incr/decr}
;;;
ArgumentExpressionList :::=
{\sffamily AssignmentExpression} , \newline
\{ \lstinline{','} , {\sffamily AssignmentExpression} \};;;
}

\subsubsection{Array subscripting}\label{subsubsec:arraysub}
\specconstraints{
One of the expressions shall have type ``pointer to object \emph{type}", the other shall have integer type, and the result has type ``\emph{type}".
}

\specsem{
The definition of the subscript operator \Verb![]! is that \Verb!E1[E2]! is identical to \Verb!(*((E1)+(E2)))!. As a consequence of the conversion rules that apply to the \Verb!+! operator, if \Verb!E1! is an array object (equivalently, a pointer to the first element of an array) and \Verb!E2! is an integer, \Verb!E1[E2]! designates the \Verb!E2!-th element of \Verb!E1! (0-indexed).
}

\subsubsection{Function calls}\label{subsubsec:fncall}
\subsubsection{Structure members}\label{subsubsec:structmem}
\subsubsection{Postfix increment and decrement operators}\label{subsubsec:postfixincrdecr}

\subsection{Unary operators}\label{subsec:unaryop}
\specsyntax{
UnaryExpression :::=
! {\sffamily PostfixExpression} 
! \lstinline{'++'} , {\sffamily UnaryExpression} // \hyperref[subsubsec:prefixincrdecr]{prefix incr}
! \lstinline{'--'} , {\sffamily UnaryExpression} // \hyperref[subsubsec:prefixincrdecr]{prefix decr}
! {\sffamily UnaryOperator} , {\sffamily CastExpression} // \hyperref[subsec:unaryop]{unary operators}
! \lstinline{'sizeof'} , \lstinline{'('} , {\sffamily TypeName} , \lstinline{')'} // \hyperref[subsubsec:sizeof]{sizeof operator}
! \lstinline{'sizeof'} , {\sffamily UnaryExpression} // \hyperref[subsubsec:sizeof]{sizeof operator}
;;;
UnaryOperator :::=
\lstinline{'\&'} // \hyperref[subsubsec:addrindirop]{address operator}
! \lstinline{'*'} // \hyperref[subsubsec:addrindirop]{indirection operator}
! \lstinline{'+'}
@ \lstinline{'-'}
@ \lstinline{'~'}
@ \lstinline{'\!'} // \hyperref[subsubsec:unaryarithop]{unary arithmetic operators}
;;;
}

\subsubsection{Prefix increment and decrement operators}\label{subsubsec:prefixincrdecr}
\subsubsection{Address and indirection operators}\label{subsubsec:addrindirop}
\subsubsection{Unary arithmetic operators}\label{subsubsec:unaryarithop}
\subsubsection{The \texttt{sizeof} operator}\label{subsubsec:sizeof}

\subsection{Cast operators}\label{subsec:cast}
\specsyntax{
CastExpression :::=
! {\sffamily UnaryExpression}
! \lstinline{'('} , {\sffamily TypeName} , \lstinline{')'} , {\sffamily CastExpression} // \hyperref[subsec:cast]{cast operator}
;;;
}

\subsection{Multiplicative operators}\label{subsec:mult}
\specsyntax{
MultiplicativeExpression :::=
{\sffamily CastExpression} , \newline
\{ ( \lstinline{'*'} @ \lstinline{'\/'} @ \lstinline{'\%'} ) , {\sffamily CastExpression} \} // \hyperref[subsec:mult]{multiplicative operators}
}

\subsection{Additive operators}\label{subsec:add}
\specsyntax{
AdditiveExpression :::=
{\sffamily MultiplicativeExpression} , \newline
\{ ( \lstinline{'+'} @ \lstinline{'-'} ) , {\sffamily MultiplicativeExpression} \}
// \hyperref[subsec:add]{additive operators}
;;;
}

\subsection{Bitwise shift operators}\label{subsec:shift}
\specsyntax{
ShiftExpression :::=
{\sffamily AdditiveExpression} , \newline
\{ ( \lstinline{'<<'} @ \lstinline{'>>'} ) , {\sffamily AdditiveExpression} \}
// \hyperref[subsec:shift]{bitwise shift operators}
;;;
}

\subsection{Relational operators}\label{subsec:rel}
\specsyntax{
RelationalExpression :::=
{\sffamily ShiftExpression} , \newline
\{ ( \lstinline{'<'} @ \lstinline{'>'} @ \lstinline{'<='} @ \lstinline{'>='} ) , {\sffamily ShiftExpression} \}
// \hyperref[subsec:rel]{relational operators}
;;;
}

\subsection{Equality operators}\label{subsec:eq}
\specsyntax{
EqualityExpression :::=
{\sffamily RelationalExpression} , \newline
\{ ( \lstinline{'=='} @ \lstinline{'\!='} ) , {\sffamily RelationalExpression} \}
// \hyperref[subsec:eq]{equality operators}
;;;
}

\subsection{Bitwise AND operator}\label{subsec:bitand}
\specsyntax{
ANDExpression :::=
{\sffamily EqualityExpression} , \{ \lstinline{'\&'} , {\sffamily EqualityExpression} \}
// \hyperref[subsec:bitand]{bitwise AND operator}
;;;
}

\subsection{Bitwise exclusive OR operator}\label{subsec:bitxor}
\specsyntax{
ExclusiveORExpression :::=
{\sffamily ANDExpression} , \{ \lstinline{'^'} , {\sffamily ANDExpression} \}
// \hyperref[subsec:bitxor]{bitwise XOR operator}
;;;
}

\subsection{Bitwise inclusive OR operator}\label{subsec:bitor}
\specsyntax{
InclusiveORExpression :::=
{\sffamily ExclusiveORExpression} , \newline
\{ \lstinline{'|'} , {\sffamily ExclusiveORExpression} \}
// \hyperref[subsec:bitor]{bitwise OR operator}
;;;
}

\subsection{Logical AND operator}\label{subsec:logand}
\specsyntax{
LogicalANDExpression :::=
{\sffamily InclusiveORExpression} , \newline
\{ \lstinline{'\&\&'} , {\sffamily InclusiveORExpression} \}
// \hyperref[subsec:logand]{logical AND operator}
;;;
}

\subsection{Logical OR operator}\label{subsec:logor}
\specsyntax{
LogicalORExpression :::=
{\sffamily LogicalANDExpression} , \newline
\{ \lstinline{'||'} , {\sffamily LogicalANDExpression} \}
// \hyperref[subsec:logor]{logical OR operator}
;;;
}

\subsection{Conditional operator}\label{subsec:cond}
\specsyntax{
ConditionalExpression :::=
{\sffamily LogicalORExpression} , \newline
[ \lstinline{'\?'} , {\sffamily Expression} , \lstinline{':'} , {\sffamily ConditionalExpression} ]
// \hyperref[subsec:cond]{conditional operator}
;;;
}

\subsection{Assignment operators}\label{subsec:assgn}
\specsyntax{
AssignmentExpression :::=
! {\sffamily UnaryExpression} , {\sffamily AssignmentOperator} , {\sffamily AssignmentExpression} // \hyperref[subsec:assgn]{assignment operators}
! {\sffamily ConditionalExpression}
;;;
AssignmentOperator :::=
! \lstinline{'='} // \hyperref[subsubsec:simpleassgn]{simple assignment}
! \lstinline{'*='}
@ \lstinline{'\/='}
@ \lstinline{'\%='}
@ \lstinline{'+='}
@ \lstinline{'-='}
@ \lstinline{'<<='}
@ \lstinline{'>>='}
@ \lstinline{'\&='}
@ \lstinline{'^='}
@ \lstinline{'|='} // \hyperref[subsubsec:compoundassgn]{compound assignment}
;;;
}

\subsubsection{Simple assignment}\label{subsubsec:simpleassgn}
\subsubsection{Compound assignment}\label{subsubsec:compoundassgn}

\subsection{Comma operator}\label{subsec:comma}
\specsyntax{
Expression :::=
{\sffamily AssignmentExpression} , \newline
\{ \lstinline{','} , {\sffamily AssignmentExpression} \} // \hyperref[subsec:comma]{comma operator}
;;;
}

