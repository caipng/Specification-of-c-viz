\newpage
\section{Memory Model}\label{sec:memory}
\spec{
The memory available to a \Verb!c-viz! program is a contiguous sequence of bytes\footnote{backed by JavaScript's \Verb!ArrayBuffer!}, each of which has a unique 32-bit address.
The lowest address is \Verb!0!, and the highest address is \texttt{\textbf{INT\_MAX}}.
}

\subsection{Memory segmentation}
\spec{
The memory is divided into four non-overlapping segments: stack, heap, data, and text.
\par
Attempting to access an address that falls outside of any of the segments causes a \emph{segmentation fault} to be thrown.
\par
\deviation{Misaligned memory access throws an error.}
}

\subsection{Stack allocations}
\spec{
Unlike typical x86 architectures, stack memory grows upwards (addresses are allocated in increasing order).
\par
The size of a stack frame is calculated by recursively scanning out all declarations made in the function body (and any blocks within).
\par
No control-flow analysis is done to reduce the size of stack frames needed. As an example, consider the following code
\begin{lstlisting}^^J
if (...) \{^^J
\ \ // block A^^J
\} else (...) \{^^J
\ \ // block B^^J
\}^^J
\end{lstlisting}
Let $S_A$ and $S_B$ denote the size of blocks A and B respectively.
Then the resulting stack frame will be of size $S_F = p_0 + S_A + p_1 + S_B$ where $p_i$ are sizes of padding bytes.
A smarter allocation algorithm will only require $S_F = \max \{p_0 + S_A, p_1 + S_B\}$ bytes.
}

\subsection{Heap allocations}\label{subsec:memmgtfns}
\spec{
\deviation{
The order and contiguity of storage allocated by successive calls to the \Verb!malloc! function is specified by a first-fit allocation algorithm.
}
\par
The pointer returned by a successful allocation is suitably aligned so that it may be assigned to a pointer to any type of object.
Specifically, the pointer contains an address with alignment \texttt{\textbf{MAX\_ALIGN}}\footnote{see \subsubsecref{subsubsec:sizealign}}.
\par
The lifetime of an allocated object extends from the allocation until the deallocation.
\par
Pointers returned are guaranteed to point to disjoint objects and contain the starting address of the allocated memory. If allocation fails, a null pointer is returned.
\par
\deviation{
If the size of memory requested is a non-positive integer, a null pointer is returned.
}
}

\subsubsection{The \texttt{malloc} function}
\spec{
\begin{lstlisting}^^J
void *malloc(int size);^^J
\end{lstlisting}
\par
The \Verb!malloc! function allocates \Verb!size! bytes of memory of indeterminate value.
\par
The \Verb!malloc! function returns a pointer to the allocated memory if allocation was successful, and a null pointer otherwise.
}

\subsubsection{The \texttt{free} function}
\spec{
\begin{lstlisting}^^J
void free(void *ptr);^^J
\end{lstlisting}
\par
The \Verb!free! function deallocates the memory pointed to by \Verb!ptr!.
\par
\deviation{
If the \Verb!ptr! argument\tablefootnote{including the case where \Verb!ptr! is a null pointer} does not match a pointer earlier returned by the \Verb!malloc! function, the interpreter throws a \Verb!"free on address not returned by malloc"! error.
}
}

\subsection{Text and data}
\spec{
The text segment is read-only and is the only segment that can be executed.
}

\subsection{Additional runtime data structures}
\subsubsection{\texttt{EffectiveTypeTable}}
\subsubsection{\texttt{InitializedTable}}
